\section{Models of Computation}
There are three models of computation used in our design,
\begin{itemize}
    \item \textbf{The addition model:} This model is used to compute the water on the surface of the windshield at each timestamp, where the newly generated "raindrop" is added to the volume of remaining water.
    \item \textbf{The FSM model:} This model is used to compute the wiper control signal. If the volume of remaining water is less than 10 ml, the FSM outputs 0 as OFF signal; if the volume of remaining water is less than 15 ml but exceeds 10 ml, the FSM outputs 1 as SLOW signal; if the volume of remaining water exceeds 15 ml, the FSM outputs 2 as FAST signal.
    \item \textbf{The multiplication model:} This model is used to compute the volume of remaining water after each wiper action. The volume of remaining water is multiplied by 1 if the wiper is OFF; the volume of remaining water is multiplied by 0.5 if the wiper works in SLOW mode; the volume of remaining water is multiplied by 0.2 if the wiper works in FAST mode.
\end{itemize}